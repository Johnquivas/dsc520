% Options for packages loaded elsewhere
\PassOptionsToPackage{unicode}{hyperref}
\PassOptionsToPackage{hyphens}{url}
%
\documentclass[
]{article}
\usepackage{amsmath,amssymb}
\usepackage{iftex}
\ifPDFTeX
  \usepackage[T1]{fontenc}
  \usepackage[utf8]{inputenc}
  \usepackage{textcomp} % provide euro and other symbols
\else % if luatex or xetex
  \usepackage{unicode-math} % this also loads fontspec
  \defaultfontfeatures{Scale=MatchLowercase}
  \defaultfontfeatures[\rmfamily]{Ligatures=TeX,Scale=1}
\fi
\usepackage{lmodern}
\ifPDFTeX\else
  % xetex/luatex font selection
\fi
% Use upquote if available, for straight quotes in verbatim environments
\IfFileExists{upquote.sty}{\usepackage{upquote}}{}
\IfFileExists{microtype.sty}{% use microtype if available
  \usepackage[]{microtype}
  \UseMicrotypeSet[protrusion]{basicmath} % disable protrusion for tt fonts
}{}
\makeatletter
\@ifundefined{KOMAClassName}{% if non-KOMA class
  \IfFileExists{parskip.sty}{%
    \usepackage{parskip}
  }{% else
    \setlength{\parindent}{0pt}
    \setlength{\parskip}{6pt plus 2pt minus 1pt}}
}{% if KOMA class
  \KOMAoptions{parskip=half}}
\makeatother
\usepackage{xcolor}
\usepackage[margin=1in]{geometry}
\usepackage{graphicx}
\makeatletter
\def\maxwidth{\ifdim\Gin@nat@width>\linewidth\linewidth\else\Gin@nat@width\fi}
\def\maxheight{\ifdim\Gin@nat@height>\textheight\textheight\else\Gin@nat@height\fi}
\makeatother
% Scale images if necessary, so that they will not overflow the page
% margins by default, and it is still possible to overwrite the defaults
% using explicit options in \includegraphics[width, height, ...]{}
\setkeys{Gin}{width=\maxwidth,height=\maxheight,keepaspectratio}
% Set default figure placement to htbp
\makeatletter
\def\fps@figure{htbp}
\makeatother
\setlength{\emergencystretch}{3em} % prevent overfull lines
\providecommand{\tightlist}{%
  \setlength{\itemsep}{0pt}\setlength{\parskip}{0pt}}
\setcounter{secnumdepth}{-\maxdimen} % remove section numbering
\ifLuaTeX
  \usepackage{selnolig}  % disable illegal ligatures
\fi
\IfFileExists{bookmark.sty}{\usepackage{bookmark}}{\usepackage{hyperref}}
\IfFileExists{xurl.sty}{\usepackage{xurl}}{} % add URL line breaks if available
\urlstyle{same}
\hypersetup{
  hidelinks,
  pdfcreator={LaTeX via pandoc}}

\author{}
\date{\vspace{-2.5em}}

\begin{document}

\hypertarget{assignment-american-community-survey-exercise}{%
\section{Assignment: American Community Survey
Exercise}\label{assignment-american-community-survey-exercise}}

\hypertarget{name-quintero-vasquez-johnatan}{%
\section{Name: Quintero Vasquez,
Johnatan}\label{name-quintero-vasquez-johnatan}}

\hypertarget{date-2023-06-25}{%
\section{Date: 2023-06-25}\label{date-2023-06-25}}

\hypertarget{load-the-ggplot2-package}{%
\subsection{Load the ggplot2 package}\label{load-the-ggplot2-package}}

library(ggplot2) theme\_set(theme\_minimal())

\hypertarget{set-the-working-directory-to-the-root-of-your-dsc-520-directory}{%
\subsection{Set the working directory to the root of your DSC 520
directory}\label{set-the-working-directory-to-the-root-of-your-dsc-520-directory}}

setwd(``C:/Users/21428899/OneDrive - Bellevue
University/Documents/GitHub/dsc520'')

\hypertarget{load-the-dataacs-14-1yr-s0201.csv-to}{%
\subsection{\texorpdfstring{Load the \texttt{data/acs-14-1yr-s0201.csv}
to}{Load the data/acs-14-1yr-s0201.csv to}}\label{load-the-dataacs-14-1yr-s0201.csv-to}}

survey\_df \textless- read.csv(``data/acs-14-1yr-s0201.csv'')

\hypertarget{what-are-the-elements-in-your-data-including-the-categories-and-data-types}{%
\section{1. What are the elements in your data (including the categories
and data
types)?}\label{what-are-the-elements-in-your-data-including-the-categories-and-data-types}}

\hypertarget{please-provide-the-output-from-the-following-functions-str-nrow-ncol}{%
\section{2. Please provide the output from the following functions:
str(); nrow();
ncol()}\label{please-provide-the-output-from-the-following-functions-str-nrow-ncol}}

str(survey\_df) nrow(survey\_df) ncol(survey\_df)

\hypertarget{create-a-histogram-of-the-hsdegree-variable-using-the-ggplot2-package.}{%
\section{3. Create a Histogram of the HSDegree variable using the
ggplot2
package.}\label{create-a-histogram-of-the-hsdegree-variable-using-the-ggplot2-package.}}

\hypertarget{set-a-bin-size-for-the-histogram.}{%
\subsection{1. Set a bin size for the
Histogram.}\label{set-a-bin-size-for-the-histogram.}}

\hypertarget{include-a-title-and-appropriate-xy-axis-labels-on-your-histogram-plot.}{%
\subsection{2. Include a Title and appropriate X/Y axis labels on your
Histogram
Plot.}\label{include-a-title-and-appropriate-xy-axis-labels-on-your-histogram-plot.}}

ggplot(survey\_df, aes(x = HSDegree)) + geom\_histogram(bins = 10) +
ggtitle(``High School Degree'') + xlab(``HS Degree'') + ylab(``GPA'')

\#4. Answer the following questions based on the Histogram produced:
\#\# 1. Based on what you see in this histogram, is the data
distribution unimodal? \#\# 2. Is it approximately symmetrical? \#\# 3.
Is it approximately bell-shaped? \#\# 4. Is it approximately normal?
\#\# 5. If not normal, is the distribution skewed? If so, in which
direction? \#\# 6. Include a normal curve to the Histogram that you
plotted. \#\# 7. Explain whether a normal distribution can accurately be
used as a model for this data.

\#5. Create a Probability Plot of the HSDegree variable.

\#6. Answer the following questions based on the Probability Plot: \# 1.
Based on what you see in this probability plot, is the distribution
approximately normal? Explain how you know. \# 2. If not normal, is the
distribution skewed? If so, in which direction? Explain how you know.

\#7. Now that you have looked at this data visually for normality, you
will now quantify normality with numbers using the stat.desc() function.
Include a screen capture of the results produced.

\#8. In several sentences provide an explanation of the result produced
for skew, kurtosis, and z-scores. In addition, explain how a change in
the sample size may change your explanation?

\end{document}
